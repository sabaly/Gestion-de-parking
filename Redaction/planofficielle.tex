\documentclass[12pt, openany]{report}
\usepackage[utf8]{inputenc}
\usepackage[T1]{fontenc}
\usepackage[a4paper,left=2cm,right=2cm,top=2cm,bottom=2cm]{geometry}
\usepackage[frenchb]{babel}
\usepackage{libertine}
\usepackage[pdftex]{graphicx}

\setlength{\parindent}{0cm}
\setlength{\parskip}{1ex plus 0.5ex minus 0.2ex}
\newcommand{\hsp}{\hspace{20pt}}
\newcommand{\HRule}{\rule{\linewidth}{0.5mm}}

\begin{document}
\begin{center}
\textit{Plan du rapport}
\end{center}
\begin{enumerate}
\item Introduction
\item Etude des cas
\item Approches Expérimental
\begin{enumerate}
\item Arduino
\begin{enumerate}


\item Étude de l' arduino

\begin{description}
\item  * Définition du module arduino
\item *  Les différents types d'arduino
\item * Le Micro-contrôleur AT-Mega62560
\item * Broches de connexion
\item  * Les ports de communications
\end{description}
\item Partie programme  Arduino

\begin{description}
\item * Environnement de Programmation

\item * Structure générale du programme arduino

\item * Injection du programme

\item * Description du programme

\item * Les étapes de téléchargement du programme

\end{description}
\item Les différents composants de notre systèmes
\begin{description}
\item *Les afficheurs LCD
\item Le buzzers
\item *Le capteurs infrarouge emetteur
\item * Le capteurs ultrasons
\item *Le module  rfid
\item *Le servo-moteurs
\end{description}
\end{enumerate}
\item  Raspberry
\begin{enumerate}
\item Etude du Raspberry-pi
\begin{description}
\item * Définition du module raspberry-pi
\item *  Les différents modèles de raspberry-pi
\item * Le raspberry pi 1.2
\item  * Les broches de connexion
\item *  Les ports de communications
\end{description}
\item Configuration du raspberry
\begin{description}
\item *Installation du raspbian
\item  *Configuration du raspbian
\item  *Environnement de programmation 
\item *Etude du langage de programmation(python)
\end{description}
\end{enumerate}
\item Liaison arduino-raspberry
\begin{enumerate}
\item configuration
\begin{description}
\item *Communication Raspberry Arduino
\item * Installation de Lamp (mysql-apache2-phpmyadmin)
\item  * configuration de lamp
\item  *mysql pour stocker les uid du rfid
\end{description}
\item Communication arduino -raspberry -interface graphique
\begin{description}
\item * Modélisation
\item * Etude du langages de programmation (php) et du framework utilisé
\item * Mise en place de l'interface graphique
\end{description}
\end{enumerate}

\end{enumerate}
\item Mise en oeuvre
\begin{description}
\item *Les afficheurs LCD
\item Le buzzers
\item *Le capteurs infrarouge emetteur
\item * Le capteurs ultrasons
\item Le module  rfid
\item Le servo-moteurs
\end{description}
\item schema du branchement final
\item Conclusion
\begin{description}
\item Bibliographie 
\item Webographie

\end{description}
\end{enumerate}
\end{document}